\section{Diskussion}
\label{sec:Diskussion}
\subsection{Fehlerrechnung}
Es gilt:
\begin{equation}
  \text{Abweichung} = \left|\frac{x_\text{theoretisch}-x_\text{praktisch}}{x_\text{theoretisch}}\right|
  \label{eqn:abweichung}
\end{equation}
\subsection{subsection name}
zur Berechnung der prozentualen Abweichung.
Durch den Versuch ergeben sich nun verschiedene Werte für die
Zeitkonstante $RC$ des Relaxionsvorgangs:
\begin{table}[H]
  \centering
  \caption{Messwerte zur Auf- und Entladung des Kondensators}
  \label{tab:ergebnis}
  \begin{tabular}{c c c}
    \toprule
    Messmethode & $RC$ & Abweichung vom Referenzwert\\
    \midrule
    Aufladung & \SI{1.388(30)e-3}{\second} & \SI{2.39}{\percent} \\
    Entladung & \SI{1.369(27)e-3}{\second} & \SI{3.73}{\percent} \\
    Amplitude & \SI{1.422(5)e-3}{\second}  & \SI{0}{\percent} \\
    Phase     & \SI{1.776(34)e-3}{\second} & \SI{24.89}{\percent} \\
  \end{tabular}
\end{table}

Als Referenzwert wird der Wert verwendet, welcher
sich durch das Auftragen der relativen Amplitude $A/U_0$ gegen die
Frequenz ergibt, da die Messmethode am wenigsten fehleranfällig ist
und dieser Wert zudem einen guten mittelwert aller Messungen ergibt.
Vergleicht man den Referenzwert mit den Werten welche sich durch
Kondensator Auf- und Entladung ergeben, so ist die Abweichung sehr gering.
Die Abweichung zwischen den beiden Messmethoden kann durch die
mangelnde Genauigkeit beim Ablesen vom Osziloskops begründet werden kann.
Der über die Phasenverschiebung bestimmte Wert weicht jedoch etwas
stärker ab, zumal die Theoriekurve in dem Polarplot eine deutlich
erkennbare Abweichung zu den Messwerten hat.
Hierbei ist jedoch zu beachten, das der Widerstand, welchen die
Erregerspannung bis zum Osziloskop zu überwinden hat mit \SI{600}{\ohm}
wesentlich geringer ist als der Widerstand von \SI{15.6}{\kilo\ohm}
welchen die Kondensatorspannung zu überwinden hat, wodurch ein leichtes
Offset der Phasen entstehen kann, welches für eine größere Abweichung sorgt.
Weiterhin ist der Fit zu den Messwerten der Phase deutlich am schlechtesten.
Insgesamt ergeben die Werte jedoch eine gute Näherung für die Zeitkonstante
des Relaxionsverhaltens an.
