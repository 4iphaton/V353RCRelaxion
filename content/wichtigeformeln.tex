\section{Wichtige Formeln für die Auswertung}
\begin{equation}
  \omega = 2\pi f
  \label{eqn:frequenz}
\end{equation}
\begin{equation}
  U(t)=U_0\left(1-e^{-t/RC}\right)
  \label{eqn:aufladen}
\end{equation}
\begin{equation}
  U(t)=U_0e^{-t/RC}
  \label{eqn:entladen}
\end{equation}
\begin{equation}
  A(\omega)=\frac{U_0}{\sqrt{1+\omega^2R^2C^2}}
  \label{eqn:amplitude}
\end{equation}
\begin{equation}
  \text{sin}(\phi)=\frac{\omega RC}{\sqrt{1+\omega^2R^2C^2}}
  \label{eqn:phase}
\end{equation}
\begin{equation}
  U_C(t)=\frac{1}{RC}\int_0^tU(t')d\, t'
  \label{eqn:integral}
\end{equation}
\begin{equation}
  \text{Abweichung} = \left|\frac{x_\text{theoretisch}-x_\text{praktisch}}{x_\text{theoretisch}}\right|
  \label{eqn:abweichung}
\end{equation}
\begin{equation}
  \phi = \increment t \cdot\omega
  \label{eqn:phasenberechnung}
\end{equation}
